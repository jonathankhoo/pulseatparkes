\documentclass{article}
\usepackage[pdftex]{graphicx}
\headheight 0in
\headsep 0in
\oddsidemargin 0in
\evensidemargin 0in
\textheight 9.5in
\textwidth 6in

\title{PULSE@Parkes Session Setup}
\author{--- Jonathan Khoo}
\date{}
\begin{document}
\maketitle

\section{Preparation}
\begin{enumerate}
\item Notify receptionist that a school is is coming and what time they are expected.
\item Collect the 4 laptops (black Dell bags), 4 monitors, and the plastic container (filled with cords, connectors, etc) from Rob Hollow's office (Room 33).
\end{enumerate}

%\includegraphics{parkes.jpg}

\section{Connecting the Laptops}
\subsection{Observing Table (inside lecture theatre)}
The observing table is in the middle of the lecture theatre. Laptops 1 and 2, and their corresponding two monitors connected via the docking stations, are situated as below:

\begin{itemize}

\item Monitors
\begin{enumerate}
\item Power cable (black connector): monitor $\Rightarrow$ power-board. (x4)
\end{enumerate}

\item Laptop 1 and Laptop 2 (each): \\
2x monitor cables (1x WHITE DVI, 1x BLUE VGA), 1x docking station, 1x power (adaptor) cable, 1x LAN cable (blue/pink), 1x keyboard, 1x mouse, 1x mousepad.
\begin{enumerate}
\item Place the laptop on top of the docking station---you should hear a click.
\item Monitor cable (white DVI ends): docking station $\Rightarrow$ monitor. (x2)
\item LAN cable (Laptop 1 --- blue cable, Laptop 2 --- pink cable): docking station $\Rightarrow$ theatre podium. Longer LAN cables can usually be found in the drawer under the podium.
\item Keyboard $\Rightarrow$ docking station (USB port).
\item Mouse $\Rightarrow$ docking station (USB port).
\item Power cable: docking station $\Rightarrow$ power-board (found in container).
\item Turn on laptop by pressing the power button on the docking station (which should be illuminated).
\item Ensure monitors are set to correct outputs. I.e., if you have connected the monitor via a DVI cable, select the DVI output mode on monitor.
\item Configure screens appropriately by right-clicking on the desktop $\Rightarrow$ "Properties" $\Rightarrow$  "Settings"
\end{enumerate}

\textbf{Note:} \\
Laptop 1 --- blue LAN cable \\
Laptop 2 --- pink LAN cable + VNC

\end{itemize}

\subsection{Data Analysis Table (outside lecture theatre)}
The table outside the lecture theatre is where the data analysis takes place with laptops 3 and 4.

\begin{itemize}
\item Laptop 3 and Laptop 4: \\
1x power cable, 1x mouse, 1x mousepad.
\begin{enumerate}
\item Power cable: laptop (back left) $\Rightarrow$ power-board.
\item Mouse: $\Rightarrow$ laptop (USB port --- right top).
\end{enumerate}

\end{itemize}

\section{Login Details for Laptops}
Username: pulse \\
Password: B0833+45 \\
Domain: (local machine) \\

\subsection*{Laptop 1 --- blue cable}
\textbf{Left Screen --- showtel}
\begin{enumerate}
\item If VNC server for showtel has not been started:
\begin{verbatim}
% ssh pulsar@pavo
pulsar@pavo's password: PULSAR_PASSWORD
% vncserver -geometry 1915x1140
% exit
\end{verbatim}

\item VNC Viewer: \\
start $\Rightarrow$ All Programs $\Rightarrow$ RealVNC $\Rightarrow$ VNC Viewer 4 $\Rightarrow$ Run VNC Viewer \\

%Server: dish0-pa:0 / Password: PULSAR\_VNC\_PASSWD $\Rightarrow$ OK
Server: dish0-pa:0 / Password: d1sh\_64m $\Rightarrow$ OK
% d1sh_64m

\item If showtel has not been started:
\begin{verbatim}
% showtel
\end{verbatim}

\end{enumerate}

\textbf{Right Screen --- TCS}
\begin{enumerate}
\item If VNC server for TCS has not been started:
\begin{verbatim}
% ssh pulsar@pavo
pulsar@pavo's password: PULSAR_PASSWORD
% vncserver -geometry 1915x1140
% exit
\end{verbatim}

\item VNC Viewer: \\
start $\Rightarrow$ All Programs $\Rightarrow$ RealVNC $\Rightarrow$ VNC Viewer 4 $\Rightarrow$ Run VNC Viewer \\
Server: pavo:\emph{[0-9+]} / Password: PULSAR\_VNC\_PASSWD $\Rightarrow$ OK

\item If TCS has not been started:
\begin{verbatim}
% TCS 
\end{verbatim}

\end{enumerate}

\subsection*{Laptop 2 --- pink cable}

\begin{enumerate}
\item VPN Client: \\
start $\Rightarrow$ All Programs $\Rightarrow$ Cisco Systems VPN Client $\Rightarrow$ VPN Client \\ \\
Connection Entry: CSIRO ATNF Marsfield \\
Host: vpn.atnf.csiro.au \\
\\
Group Authentication: \\

%Name: Pulse at Parkes / Password: VPN\_GROUP\_PASSWD \\
Name: Pulse at Parkes / Password: Vela \\
% Vela

%Username: pulsar / Password: VPN\_PASSWD
Username: pulsar / Password: PSR0833-45
% PSR0833-45

\end{enumerate}


\textbf{Left Screen --- Display the pulse-profile and the chat log}
\begin{enumerate}
\item Firefox $\Rightarrow$ http://outreach.atnf.csiro.au/education/pulseatparkes/student\_observer.html
\end{enumerate}

\textbf{Right Screen --- Skype and webcam}
\begin{enumerate}
\item Attach the webcam to the top of the monitor, facing the current observing students.
\item Webcam $\Rightarrow$ docking station (USB Port).
\item Left side of the screen: Run Skype. Video call (Skype account: Username: pulseatparkes-observer / Password: Vela) with astronomer (pulseatparkes-astronomer) at Parkes.
\item Right side of the screen: display the webcam of the Dish.
\begin{enumerate}
\item Firefox $\Rightarrow$ \emph{http://pkswebcam01.atnf.csiro.au:8080}
\item Video Source $\Rightarrow$ \emph{Channel : 1}
\item Video Size $\Rightarrow$ \emph{Large}
\end{enumerate}
\end{enumerate}

\subsection*{Laptop 3 and Laptop 4 --- Wireless Network Connection}
\begin{enumerate}
\item Enable wireless connection: Laptop (left - middle) wireless switch.
\item Right click $\Rightarrow$ Wireless icon (bottom - right) $\Rightarrow$ Open Intel PROSet/Wireless
\item Select network: $<$SSID not broadcast$>$ where its authentication method (right click $\Rightarrow$ properties) is: WPA2-Enterprise.
\item Enter (NEXUS) username and password appropriately.
\item Firefox $\Rightarrow$ \emph{http://pulseatparkes.atnf.csiro.au/distance}
\end{enumerate}


\section{Projectors}
Touch theatre screen to turn it on.

\textbf{Left Projector --- Laptop}
\begin{enumerate}
\item LAN Cable: Laptop (back right) $\Rightarrow$ desk/podium.
\item Monitor Cable $\Rightarrow$ Computer VGA Port (desk/podium).
\item Power Cable: Laptop (back left) $\Rightarrow$ Power Socket (desk/podium).

\item VPN Client: \\
start $\Rightarrow$ All Programs $\Rightarrow$ Cisco Systems VPN Client $\Rightarrow$ VPN Client \\ \\
Connection Entry: CSIRO ATNF Marsfield \\
Host: vpn.atnf.csiro.au \\
\\
Group Authentication: \\
Name: Pulse at Parkes / Password: Vela
% Vela
Username: pulsar / Password: PSR0833-45
% PSR0833-45

\begin{itemize}
\item Right-click $\Rightarrow$ \emph{Graphics Options} $\Rightarrow$ \emph{Display} $\Rightarrow$ \emph{Notebook + Monitor}
\item Right-click $\Rightarrow$ \emph{Graphics Properties} $\Rightarrow$ \emph{Display Settings} $\Rightarrow$ \emph{Screen Resolution} $\Rightarrow$ \emph{1900x1200}
\end{itemize}

On the Theatre touch-screen:
\begin{itemize}
\item \emph{Video} $\Rightarrow$ \emph{Centre Projector} $\Rightarrow$ \emph{Centre VGA}
\end{itemize}

\item VNC Viewer --- TCS: \\
start $\Rightarrow$ All Programs $\Rightarrow$ RealVNC $\Rightarrow$ VNC Viewer 4 $\Rightarrow$ Run VNC Viewer \\
Server: [pavo$|$orion]:\emph{[0-9+]} / Password: PULSAR\_VNC\_PASSWD 

\item VNC Viewer --- Showtel: \\
start $\Rightarrow$ All Programs $\Rightarrow$ RealVNC $\Rightarrow$ VNC Viewer 4 $\Rightarrow$ Run VNC Viewer \\
VNC Viewer $\Rightarrow$ Server: [pavo$|$orion]:\emph{[0-9+]} / Password: PULSAR\_VNC\_PASSWD 
\end{enumerate}

\begin{description}
\item[Right Projector] --- Theatre PC
\item \emph{Video} $\Rightarrow$ \emph{Right Projector} $\Rightarrow$ \emph{Theatre PC}
\item \emph{Home} $\Rightarrow$ expand the target computer $\Rightarrow$ login (NEXUS domain)
\begin{enumerate}
\item Display the Parkes webcam in the upper section of the screen.
\begin{itemize}
\item Firefox $\Rightarrow$ \emph{http://pkswebcam01.atnf.csiro.au:8080}
\item \emph{Video Source} $\Rightarrow$ \emph{Channel : 1}
\item \emph{Video Size} $\Rightarrow$ \emph{Large}
\end{itemize}
\item Display pulse-profile and chat on the lower section of the screen.
\begin{itemize}
\item Firefox $\Rightarrow$ \emph{http://outreach.atnf.csiro.au/education/pulseatparkes/student\_observer.html}
\end{itemize}


\end{enumerate}

\end{description}
Shutting down:
\begin{itemize}
\item \emph{Video} $\Rightarrow$ \emph{Centre Projector} $\Rightarrow$ \emph{Power off}
\item \emph{Video} $\Rightarrow$ \emph{Right Projector} $\Rightarrow$ \emph{Power off}
\end{itemize}

\section{Astronomer @ Parkes}
\begin{enumerate}
\item Open chat window as astronomer.
\begin{itemize}
\item Firefox $\Rightarrow$ \emph{http://outreach.atnf.csiro.au/education/pulseatparkes/astronomer\_login.html}
\end{itemize}
\end{enumerate}

%\section{Computers One (Left side)}
\section{Miscellaneous}
\subsection*{Adding a new school}
\begin{enumerate}
\item Open \emph{/nfs/wwwresearch/pulsar/pulseATpks/session}
\item Retaining the file's format, append: \\
\\
ID: XXXX \\
DATE: DD$-$MMM$-$08
================================ \\
\end{enumerate}

\subsection*{Clearing the chat log}
In the (usual) case of spam, the chat log for the student\_observer.html page must be cleared.
\begin{verbatim}
% ssh pulsar@atlas
pulsar@atlas's password: PULSAR_PASSWD
% cd /export/www/vhosts/outreach/htdocs/education/pulseatparkes
% mv session_1.log [filename]
% touch session_1.log
% exit
\end{verbatim}

\subsection*{Starting background script on Lagavulin}
\begin{verbatim}
% ssh pulsar@lagavulin
pulsar@lagavulin's password: PULSAR_PASSWD
% cd /psr1/pulseATpks
% ./pap.csh [school number]
\end{verbatim}


\subsection*{Printouts}
For each pulsar observation, these files are produced and transferred to \emph{/nfs/wwwresearch/pulsar/pulseATpks/}: \\
\\
JXXXX$[-|+]$XXXX.[\emph{Group ID}].[\emph{Observation Number}].4channels.txt \\
JXXXX$[-|+]$XXXX.[\emph{Group ID}].[\emph{Observation Number}].8channels.txt \\
JXXXX$[-|+]$XXXX.[\emph{Group ID}].[\emph{Observation Number}].gif \\
JXXXX$[-|+]$XXXX.[\emph{Group ID}].[\emph{Observation Number}].ps \\
\\
Printouts of the pulse-profiles are given to each student. This command will print the pulse-profile to the printer located in room 82 : \\
\\
lpr -PEPPI-B1LG-82-HP4200 JXXXX$[-|+]$XXXX.\emph{[Group ID]}.\emph{[Observation Number]}.ps


\subsection*{Skype Accounts}
Two Skype accounts are used for the video-link between the observing students and the astronomer at Parkes:
\begin{itemize}
\item pulseatparkes-astronomer / Vela
\item pulseatparkes-observer / Vela
\end{itemize}

\subsection*{Theatre Lights}
The controls for the lights is located at the entrance closest to reception. Slide the bars up/down to the desired level.

\subsection*{Troubleshooting}
\subsubsection*{Background script not running}

Symptom: no ticks are appearing next to pulsars in the Student Data Archive. If (for whatever reason) the P@P background script has not been running during the P@P session, these steps should be followed to ensure the P@P data gets processed and copied to Epping so that the students can use the P@P modules with their data.

\begin{enumerate}
\item Determine school identifier - go to the Student Data Archive and note the current number.
\item Determine P@P observations.
\begin{verbatim}
$ hostname
lagavulin
$ ls -lrt /nfs/PKCCC3\_1/*.rf
\end{verbatim}

Make a note of which files have been observed during the P@P session (by time).

\item Check for multiple observations of the same pulsar.
\begin{verbatim}
\$ vap -c name <list of P@P pulsars>
\end{verbatim}

\item Run finishObs (script to process the data and transform it into the format required for the P@P modules).
\end{enumerate}

\begin{verbatim}
$ cd /psr1/pulseATpks
$ ./finishObs <filename> <session identifier> <observation number>
\end{verbatim}
E.g. if /nfs/PKCCC3\_1/s123456\_789012.rf is the 2nd observation of pulsar J1234-5678, observed by school 21, then the finishObs command is:
\begin{verbatim}
$ ./finishObs /nfs/PKCCC3\_1/s123456\_789012.rf 21 2
\end{verbatim}

\end{document}
